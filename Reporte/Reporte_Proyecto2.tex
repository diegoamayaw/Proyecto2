
%% Reporte_Proyecto1.tex
%% 31/08/2017
%% Julio Ernesto Sánchez Díaz CU: 148221
%% Diego Amaya Willhelm CU: 149119 
%% Gumer Israel Rodríguez Martínez CU: 149109
%% Gabriel Reynoso Romero CU: 150904

%
\documentclass[journal]{IEEEtran}
\usepackage{blindtext}
\usepackage{graphicx}
\usepackage[pdftex]{graphicx}
\usepackage{algorithmic}
\usepackage{mdwmath}
\usepackage{mdwtab}
\usepackage{url}

% correct bad hyphenation here
\hyphenation{op-tical net-works semi-conduc-tor}


\begin{document}
%
% paper title
% can use linebreaks \\ within to get better formatting as desired
\title{Proyecto 2: Control Cinematico de un Robot}
%
%
% author names and IEEE memberships
% note positions of commas and nonbreaking spaces ( ~ ) LaTeX will not break
% a structure at a ~ so this keeps an author's name from being broken across
% two lines.
% use \thanks{} to gain access to the first footnote area
% a separate \thanks must be used for each paragraph as LaTeX2e's \thanks
% was not built to handle multiple paragraphs
%

\author{Julio~Ernesto~Sanchez~Diaz,~\ITAM{Ing. en Mecatronica,~148221}
        Diego~Amaya~Willhelm,~\ITAM{Ing. en Mecatronica,~149119}
        Gumer~Israel~Rodríguez~Martínez,~\ITAM{Ing. en Mecatrónica y telecomunicaciones,~149109}
        Gabriel~Reynoso~Romero,~\ITAM{Ing. en Mecatrónica e industrial,~150904}
        }% <-this % stops a space

% The paper headers
\markboth{Reporte en \LaTeX\ del Proyecto 2 de Robotica, Noviembre~2017}%
{Shell \MakeLowercase{\textit{et al.}}: Formato IEEEtran.cls for Journals}
% The only time the second header will appear is for the odd numbered pages
% after the title page when using the twoside option.
% 
% *** Note that you probably will NOT want to include the author's ***
% *** name in the headers of peer review papers.                   ***
% You can use \ifCLASSOPTIONpeerreview for conditional compilation here if
% you desire.




% If you want to put a publisher's ID mark on the page you can do it like
% this:
%\IEEEpubid{0000--0000/00\$00.00~\copyright~2007 IEEE}
% Remember, if you use this you must call \IEEEpubidadjcol in the second
% column for its text to clear the IEEEpubid mark.



% use for special paper notices
%\IEEEspecialpapernotice{(Invited Paper)}




% make the title area
\maketitle


\begin{abstract}
En este proyecto se busca generar trayectorias usando un controlador cinematico visto en clase para un robot tipo Ackerman y el simulador Gazebo para recibir el estado actual del robot y controlarlo. 
\end{abstract}
% IEEEtran.cls defaults to using nonbold math in the Abstract.
% This preserves the distinction between vectors and scalars. However,
% if the journal you are submitting to favors bold math in the abstract,
% then you can use LaTeX's standard command \boldmath at the very start
% of the abstract to achieve this. Many IEEE journals frown on math
% in the abstract anyway.

% Note that keywords are not normally used for peerreview papers.
\begin{IEEEkeywords}
IEEEtran, Reporte, Robotica, ROS.
\end{IEEEkeywords}






% For peer review papers, you can put extra information on the cover
% page as needed:
% \ifCLASSOPTIONpeerreview
% \begin{center} \bfseries EDICS Category: 3-BBND \end{center}
% \fi

% For peerreview papers, this IEEEtran command inserts a page break and
% creates the second title. It will be ignored for other modes.
\IEEEpeerreviewmaketitle



\section{Introduccion}
El uso de controladores en un sistema retroalimentado, en este caso un robot, nos ayuda a lograr un movimiento deseado de forma precisa, dependiendo del controlador. En este caso se busca controla el movimiento de un robot, por lo que buscamos que dado un estado final, el robot logre trasladarse a dicho punto. El controlador usado requiere de la pose actual del robot como retroalimentacion, y mediante operaciones matematicas se logra calcular la velocidad y el angulo del volante, las cuales se transmiten a los respectivos actuadores (motor y volante). Estos valores se deben de publicar en un topico que pueda leer el programa de Gazebo para este lo pueda representar en el programa y asi poder publicar en otro topico la pose actual del robot, topico al cual debe de suscribirse el nodo controlador para retroalimentar los datos. 
\subsection{Marco teorico}
La cinematica de un robot nos permite estudiar los movimientos de un robot con respecto a un sistema de referencia. En un analisis cinematico la posicion, velocidad y aceleracion de cada uno de los elementos del robot son calculados sin considerar las fuerzas que causan el movimiento con apoyo a las propiedades geometricas del robot y de su entorno basadas en el tiempo de movimiento. 

El control de un robot es un sistema que nos permite converger a un valor deseado retroalimentando valores al sistema por parte de sensores u otras herramientas para en caso de presentar errores con respecto al valor deseado, hacer modificaciones en los actuadores a partir de modelos matematicos para así acercarse al valor deseado en cada ciclo. 

La simulacion de un robot es parte esencial de las herramientas de programacion de cualquier usuario de ROS. Gazebo es un simulador que nos permite de forma rapida correr y probar algoritmos, diseños de robots, comportamiento y hasta entrenamiento de inteligencia artificial usando escenarios realistas gracias a su motor grafico que nos permite observar y analizar los resultados.

\section{Experimentos}

% needed in second column of first page if using \IEEEpubid
%\IEEEpubidadjcol

% An example of a floating figure using the graphicx package.
% Note that \label must occur AFTER (or within) \caption.
% For figures, \caption should occur after the \includegraphics.
% Note that IEEEtran v1.7 and later has special internal code that
% is designed to preserve the operation of \label within \caption
% even when the captionsoff option is in effect. However, because
% of issues like this, it may be the safest practice to put all your
% \label just after \caption rather than within \caption{}.
%
% Reminder: the "draftcls" or "draftclsnofoot", not "draft", class
% option should be used if it is desired that the figures are to be
% displayed while in draft mode.
%
%\begin{figure}[!t]
%\centering
%\includegraphics[width=2.5in]{myfigure}
% where an .eps filename suffix will be assumed under latex, 
% and a .pdf suffix will be assumed for pdflatex; or what has been declared
% via \DeclareGraphicsExtensions.
%\caption{Simulation Results}
%\label{fig_sim}
%\end{figure}

% Note that IEEE typically puts floats only at the top, even when this
% results in a large percentage of a column being occupied by floats.


% An example of a double column floating figure using two subfigures.
% (The subfig.sty package must be loaded for this to work.)
% The subfigure \label commands are set within each subfloat command, the
% \label for the overall figure must come after \caption.
% \hfil must be used as a separator to get equal spacing.
% The subfigure.sty package works much the same way, except \subfigure is
% used instead of \subfloat.
%
%\begin{figure*}[!t]
%\centerline{\subfloat[Case I]\includegraphics[width=2.5in]{subfigcase1}%
%\label{fig_first_case}}
%\hfil
%\subfloat[Case II]{\includegraphics[width=2.5in]{subfigcase2}%
%\label{fig_second_case}}}
%\caption{Simulation results}
%\label{fig_sim}
%\end{figure*}
%
% Note that often IEEE papers with subfigures do not employ subfigure
% captions (using the optional argument to \subfloat), but instead will
% reference/describe all of them (a), (b), etc., within the main caption.


% An example of a floating table. Note that, for IEEE style tables, the 
% \caption command should come BEFORE the table. Table text will default to
% \footnotesize as IEEE normally uses this smaller font for tables.
% The \label must come after \caption as always.
%
%\begin{table}[!t]
%% increase table row spacing, adjust to taste
%\renewcommand{\arraystretch}{1.3}
% if using array.sty, it might be a good idea to tweak the value of
% \extrarowheight as needed to properly center the text within the cells
%\caption{An Example of a Table}
%\label{table_example}
%\centering
%% Some packages, such as MDW tools, offer better commands for making tables
%% than the plain LaTeX2e tabular which is used here.
%\begin{tabular}{|c||c|}
%\hline
%One & Two\\
%\hline
%Three & Four\\
%\hline
%\end{tabular}
%\end{table}


% Note that IEEE does not put floats in the very first column - or typically
% anywhere on the first page for that matter. Also, in-text middle ("here")
% positioning is not used. Most IEEE journals use top floats exclusively.
% Note that, LaTeX2e, unlike IEEE journals, places footnotes above bottom
% floats. This can be corrected via the \fnbelowfloat command of the
% stfloats package.



\section{Conclusiones}
La parte que mas nos causó problemas fue la instalacion de Gazebo, debido a que los errores que manda al ejecutar las instrucciones de github no contienen mucha informacion y la informacion que contiene es muy tecnica, por lo que necesitamos ayuda para resolver esos problemas. Una vez resuelto ese problema, la programacion del controlador no tuvo tampoco mucha dificultad, debida a que solo implementammos un codigo preexistente y previamente probado, sin embargo nos encontramos con ciertas dificultades a la hora de probarlo, ya que el coche en el simulador no respondia como se esperaba. Nos dimos cuenta que el error se encontraba en errores de nuestro codigo, y no del simulador.
Gracias al simulador pudimos ver el efecto que tenia cambiar ciertos parametros del programa como por ejemplo los valores de K del controlador, pudimos observar como el coche no convergia al punto deseado cuando asignabamos valores de K que no cumplian con las condiciones que se mencionaron en las instrucciones y tambien pudimos comprobar por segunda ocasion que se necesita mejorar la implementacion de ROS o de los paquetes de ROS para que puedan funcionar los programas mejor a frecuencias mas altas, lo que nos permitiria una mayor precision y velocidad de los programas, ya que el hardware de las computadoras actuales nos lo permiten.




% if have a single appendix:
%\appendix[Proof of the Zonklar Equations]
% or
%\appendix  % for no appendix heading
% do not use \section anymore after \appendix, only \section*
% is possibly needed

% use appendices with more than one appendix
% then use \section to start each appendix
% you must declare a \section before using any
% \subsection or using \label (\appendices by itself
% starts a section numbered zero.)
%


%\appendices
%\section{Proof of the First Zonklar Equation}
%Some text for the appendix.

% use section* for acknowledgement
%\section*{Acknowledgment}


%The authors would like to thank...


% Can use something like this to put references on a page
% by themselves when using endfloat and the captionsoff option.
%\ifCLASSOPTIONcaptionsoff
%  \newpage
%\fi



% trigger a \newpage just before the given reference
% number - used to balance the columns on the last page
% adjust value as needed - may need to be readjusted if
% the document is modified later
%\IEEEtriggeratref{8}
% The "triggered" command can be changed if desired:
%\IEEEtriggercmd{\enlargethispage{-5in}}

% references section

% can use a bibliography generated by BibTeX as a .bbl file
% BibTeX documentation can be easily obtained at:
% http://www.ctan.org/tex-archive/biblio/bibtex/contrib/doc/
% The IEEEtran BibTeX style support page is at:
% http://www.michaelshell.org/tex/ieeetran/bibtex/
%\bibliographystyle{IEEEtran}
% argument is your BibTeX string definitions and bibliography database(s)
%\bibliography{IEEEabrv,../bib/paper}
%
% <OR> manually copy in the resultant .bbl file
% set second argument of \begin to the number of references
% (used to reserve space for the reference number labels box)
\begin{thebibliography}{1}

\bibitem{IEEEhowto:kopka}
Jason M. O’Kane, \emph{A gentle introduction to \ROS}, 2nd~ed.\hskip 1em plus
  0.5em minus 0.4em\relax University of South Carolina, USA:  Jason Matthew O’Kane, 2014.

\end{thebibliography}

% biography section
% 
% If you have an EPS/PDF photo (graphicx package needed) extra braces are
% needed around the contents of the optional argument to biography to prevent
% the LaTeX parser from getting confused when it sees the complicated
% \includegraphics command within an optional argument. (You could create
% your own custom macro containing the \includegraphics command to make things
% simpler here.)
%\begin{biography}[{\includegraphics[width=1in,height=1.25in,clip,keepaspectratio]{mshell}}]{Michael Shell}
% or if you just want to reserve a space for a photo:

%\begin{IEEEbiography}[{\includegraphics[width=1in,height=1.25in,clip,keepaspectratio]%{picture}}]{John Doe}
%\blindtext
%\end{IEEEbiography}

% You can push biographies down or up by placing
% a \vfill before or after them. The appropriate
% use of \vfill depends on what kind of text is
% on the last page and whether or not the columns
% are being equalized.

%\vfill

% Can be used to pull up biographies so that the bottom of the last one
% is flush with the other column.
%\enlargethispage{-5in}



% that's all folks
\end{document}


